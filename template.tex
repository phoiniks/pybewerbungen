\documentclass[a4paper,11pt]{dinbrief} 
\usepackage[utf8]{inputenc}
\usepackage[greek.polutoniko, ngerman]{babel}
\usepackage[autostyle=true,german=quotes]{csquotes}
\setlength{\headheight}{1.1\baselineskip}
\usepackage{scrdate}
\usepackage{footnote}
\usepackage{scrlayer-scrpage}
\setlength{\headsep}{10mm}
\pagestyle{scrheadings}
\chead{\textgreek{>En >arq\~h| \~>hn <o l'ogoc}}
\pagenumbering{gobble}
\hyphenation{ Office }
\begin{document}
\subject{ {{ bezeichnung }} }
\backaddress{Andreas Grell,
  Kühnehöfe 25,
  D-22761 Hamburg
} 
\signature{Andreas Grell} 
\Datum{\todaysname, \today} 
\address{Andreas Grell \\
  Hispanist, Germanist, Softwareentwickler \\
  Kühnehöfe 25 \\
  D-22761 Hamburg} 
\begin{letter}
    { {{ firma|safe }} \\
    
      z.Hd. {{- ' ' + anrede }}n {{- ' ' + ansprechpartner }} \\
    
    
      z.Hd.{{- ' ' + anrede }} {{- ' ' + ansprechpartner }} \\
    
    {{ strasse }} \\ 
    {{ ort }} }
    % Dies ist übrigens ein Befehl: \begin{letter}{Adresse}
    
      \opening{ Sehr geehrter {{- ' ' + anrede }} {{- ' ' + ansprechpartner }},}
%%          
    
      \opening{ Sehr geehrte {{- ' ' + anrede }} {{- ' ' + ansprechpartner }},}
    
    \opening{ Sehr geehrte Damen und Herren, }            
      
                

      durch  eure  Ihre  Stellenanzeige bei {{ quelle }} auf
      {{ firma }} aufmerksam geworden, möchte ich mich mit dem vorliegenden Anschreiben als \enquote{ {{- bezeichnung -}} }
      bei  euch  Ihnen  bewerben.
      
      Begonnen habe ich, nach einem als C/C++-Programmierer unter Linux abgelegten Zertifikat, bei HAVI Solutions
      GmbH \& Co. KG, einem Hamburger Spezialisten für Datenmigration, als C-, Perl- und Bash-Programmierer, bei
      dem ich mich sowohl mit der Transformation von SQL-Quelldatenstrukturen (MySQL) in ebensolche Zieldatenstrukturen
      als auch der Extraktion von Bilddaten aus obsoleten Datencontainerformaten auseinandersetzte. Um die besagten
      Daten aus den mit herkömmlicher Software nicht mehr zugänglichen Containern zu extrahieren, schrieb ich in
      C eigene Programme. Nach der Extraktion verarbeitete ich die Bilddaten mit eigenen Python-Programmen weiter.

      Nach einem kurzen Intermezzo mit Django, PostgreSQL und Python bei der Firma nachtblau tv, arbeitete ich bei dem
      Börsenportal ARIVA in Kiel an der PDF-Erstellung von Produktinformationsblättern mit Hilfe von
      Perl. Hierbei kam mir meine Vorliebe für reguläre Ausdrücke zugute, mit deren Hilfe ich die Korrektheit
      der immer wieder notwendigen Änderungen an den auszuliefernden Produktinformationen gewährleistete.
      
      Daran schloss sich eine Projektarbeit an einem webbasierten Office-System bei der Hannoveraner Firma
      Delticom an, in deren Rahmen ich eine Google-Suggest-artige Funktionalität auf der Grundlage von Perl und AJAX
      entwickelte.

      Zuletzt wirkte ich bei DSS in Stockelsdorf an der Funktionserweiterung der Aboverwaltung mit. Die verwendeten
      Programmiersprachen waren hier Perl und COBOL, das ich eigens für die von mir zu bewältigenden Aufgaben erlernte.
      Als Datenbanksystem wurde dort die Berkeley Database verwendet.

      Nach einem von Dezember 2019 bis Februar 2020 absolvierten Lehrgang verfüge ich auch über Kenntnisse in Java.
      Nachdem ich mir bereits während des Lehrgangs die Grundlagen in Swing, JDBC und Maven im Selbststudium angeeignet
      habe, beschäftige ich mich momentan auch mit JPA und Hibernate. Darüber hinaus interessiere ich mich unter
      Verwendung der Programmiersprache Python für Natural Language Processing und neuronale Netze -- zwei
      Technologien, in die ich mich gerade einarbeite.

      Als Sprach- und Literaturwissenschaftler sowie altgedienter Schlussredakteur mit jahrzehntelanger
      Redaktionserfahrung bei diversen Zeitschriftenverlagen (DER SPIEGEL, TV Movie, TV Spielfilm, PRINZ etc.)
      verfüge ich zudem über eine ausgeprägte Kommunikationsstärke (sowohl auf Deutsch als auch auf Spanisch
      und Englisch) und die Fähigkeit, selbst komplexe Sachverhalte allgemeinverständlich sowohl schriftlich als
      auch mündlich zu vermitteln. Als Text-Layout-System verwende ich mittlerweile bevorzugt \LaTeXe, weil es Texte
      akkurater formatiert als die gängigen Office-Anwendungen.

      Verfügbar bin ich ab sofort für ein Bruttojahresgehalt von {{ gehalt }} Euro.

      Unter https://github.com/phoiniks  könnt ihr  können Sie  im
      Verlauf von Projekten entstandene Codeschnipsel besichtigen. Sollten  euch 
      Ihnen  mein hier freilich nur grob umrissenes Profil und das erwähnte Softwarematerial zusagen, so würde ich
      mich freuen, persönlich mit  euch  Ihnen  über das oben genannte
      Angebot zu sprechen.

      
        \closing{ Liebe Grüße }
      
        \closing{ Mit freundlichen Grüßen }
      

\par
\encl{Lebenslauf und Zeugnisse} 
\end{letter} 
\end{document} 
